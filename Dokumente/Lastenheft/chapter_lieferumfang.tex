\chapter{Lieferumfang und Abnahmekriterien}

\section{Lieferumfang}
Zum Lieferumfang unseres Teams gehört zum einen das GUI basierte Tool zum Einlesen, Anzeigen und Bearbeiten von Roboterfußball-Videos, sowie deren synchronisierte Darstellung im GUI. Dabei sollen auch die dazugehörigen Log-Files eingelesen werden können und mit den Videos synchronisiert werden.
\\Zum anderen enthält der Lieferumfang auch die ausführliche Dokumentation unseres Projektes. 
So wird zur Abgabe ein Javadoc zur Codedokumentation abgegeben. 
Des weiteren wird ein Handbuch eingereicht, welches eine detaillierte Anleitung zur Handhabung der gelieferten Software, sowie eine Entwurfsbeschreibung, unter anderem in Form eines UML-Diagrammes, enthält.
Schließlich gehöhrt auch ein Installationshandbuch zum Lieferumfang. 

\section{Abnahmekriterien}
Das wichtigste Abnahmekriterium stellt die Einhaltung des Lieferumfanges dar. 
Dabei sollen alle Pflichtanforderungen erfüllt und eingehalten werden. Zusätzlich ist vorgesehen, soviele optionale Anforderungen wie möglich umzusetzen, sich dabei vorerst aber vor allem auf die Kerninhalte, unter anderem das Schneiden der Videos, zu konzentrieren.\\
Die Videos müssen synchronisiert abspielbar sein, ein framegenaues agieren und modifizieren soll gewährleistet sein. Des weiteren muss eine Audiospur auswählbar sein, um einen durchgängigen Ton zu gewährleisten. Diese wird ebenfalls mit den Videos synchronisiert.