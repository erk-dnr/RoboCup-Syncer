\chapter{Qualitätsmatrix nach ISO 25010}

\begin{table}[!ht]
\resizebox{\columnwidth}{!}{\begin{tabular}{|c||c|c|c|c|c|c|c|c|}
\hline
\multicolumn{1}{|l||}{\textbf{Wichtigkeit}} & \multicolumn{1}{l|}{\textbf{\begin{tabular}[c]{@{}l@{}}Funktionale \\ Tauglichkeit\end{tabular}}} & \multicolumn{1}{l|}{\textbf{Effizienz}} & \multicolumn{1}{l|}{\textbf{\begin{tabular}[c]{@{}l@{}}Kompati-\\ bilität\end{tabular}}} & \multicolumn{1}{l|}{\textbf{Nutzbarkeit}} & \multicolumn{1}{l|}{\textbf{\begin{tabular}[c]{@{}l@{}}Zuverlässig-\\ keit\end{tabular}}} & \multicolumn{1}{l|}{\textbf{Sicherheit}} & \multicolumn{1}{l|}{\textbf{Wartbarkeit}} & \multicolumn{1}{l|}{\textbf{\begin{tabular}[c]{@{}l@{}}Übertrag-\\ barkeit\end{tabular}}} \\ \hline
Hoch & X &  &  &  &  &  & X &   \\ \hline
Mittel &  & X & X & X & X &  &  & X \\ \hline
Niedrig &  &  &  &  &  &  &  &  \\ \hline
N.A. &  &  &  &  &  & X &  &  \\ \hline
\end{tabular}}
\end{table}
\section{Zu den Qualitätsparametern:}

\textbf{\underline {Funktionale Tauglichkeit:}}
(Funktionale Vollständigkeit/Korrektheit/Eignung) \\
Vollständig funktionierendes, den funktionalen Anforderungen entsprechendes Endprodukt $\Rightarrow$ hohe Priorität\\
\noindent
\textbf{\underline {Effizienz:}}
(Zeitverhalten, Ressourcennutzung, Kapazität)\\
Tolerierbare Lauf-/Antwortzeiten und Ressourcenauslastung $\Rightarrow$ mittlere Priorität\\
\noindent
\textbf{\underline {Kompatibilität:}}
(Koexistenz, Interoperabilität)\\
Unhinderliche Koexistenz mit anderer Software; Informationsaustausch zwischen Modulen $\Rightarrow$ mittlere Priorität\\
\noindent
\textbf{\underline {Nutzbarkeit:}}
(Lernbarkeit, Bedienbarkeit, Vorbeugung von Nutzerfehlern, Gestaltung des Nutzerinterfaces, Zugänglichkeit)\\
Gestaltung des Produkts für (Hoch-)Spezialisierte Anwendergruppe; Übersichtliche, intuitive Bedienung $\Rightarrow$ mittlere Priorität\\
\noindent
\textbf{\underline{Zuverlässigkeit:}}
(Ausgereiftheit, Verfügbarkeit, Fehlertoleranz, Wiederherstellbarkeit)\\
Stabile Laufzeit/Fehlerbehandlung; gehobene Wiederherstellbarkeit als nice-to-have $\Rightarrow$ mittlere Priorität\\
\noindent
\textbf{\underline {Sicherheit:}}
(Diskretion, Integrität, Unverfolgbarkeit, Verantwortlichkeit, Authentizität)\\
Irrelevant/Nicht gefordert $\Rightarrow$ nicht anwendbar\\
\noindent
\textbf{\underline {Wartbarkeit:}}
(Modularität, Wiederverwendbarkeit, Analysierbarkeit, Modifizierbarkeit, Testbarkeit)\\
Modularität/Modifizierbarkeit gefordert $\Rightarrow$ hohe Priorität\\
\noindent
\textbf{\underline {Übertragbarkeit:}}
(Adaptivität, Installierbarkeit, Ersetzbarkeit)\\
Plattformunabhängigkeit gefordert $\Rightarrow$ mittlere Priorität

