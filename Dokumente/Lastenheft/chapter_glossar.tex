\chapter{Glossar}
\begin{description}
\item [Java] Java ist eine objektoriente Programmiersprache, die sich durch ihre Plattformunabhängigkeit auszeichnet, der Quellcode wird in bytecode kompiliert, der dann auf jeder JVM unabhängig von System und Architektur laufen kann.
\item [XML] (engl.: \textsf{Extensible Markup Language}) ist eine Auszeichnungssprache, die hierarchisch strukturierte Daten darstellen kann. Es wird auch eingesetzt für den plattform- und implementationsunabhängigen Datenaustausch.
\item [Parser] Ein Parser ist ein Programm, das eine Eingabe derart zerlegt und umwandelt, dass es in einem, für die Weiterverarbeitung, (optimalen) Format ist.
\item [API] Eine Schnittstelle zur Anwendungsprogrammierung, häufig kurz API genannt (engl.: \textsf{application prgramming interface}, wörtl.: \textsf{Anwendungsprogrammierschnittstelle}), über das andere Programme sich an das Softwaresystem anbinden kann.
\item [Library] Als Library (dt.: \textsf{Programmierbibliothek}; kurz: \textsf{Bibliothek}) bezeichnet man eine Sammlung von Unterprogrammen/-Routinen inklusive, falls vorhanden, Dokumentation, templates, pre-written code, [...]. Eine Library bietet Lösungswege/-ansätze für thematisch zusammenhängende Problemstellungen. Der Zugriff auf eine Bibliothek erfolgt über eine API, die sich aus der Gesamtheit der öffentlichen Funktionen und Klassen zusammensetzt.
\item [GUI] (engl.: \textsf{graphical user interface}) ist eine grafisches Benutzeroberfläche, die als Benutzerschnittstelle dient, indem einerseits der Mensch mit dem Programm kommunizieren kann, in Form von Buttons, Slidern, o.Ä., andererseits aber auch das Programm mit dem Menschen kommunizieren kann, indem Daten auf der GUI visualisiert werden.
\item [Frame] Einzelbild in Filmen, Animationen und Computerspielen
\item [FPS] Frames pro Sekunde
\item [Synchronisation] bedeutet in diesem Zusammenhang mehrere Elemente zeitlich auf dem Wiedergabegerät an eine Datenquelle anzupassen.
\item [Software] ist ein Computerprogramm, einschließlich ihrer dazugehörigen Dokumentation und Konfigurationen, die erforderlich sind um das Programm auszuführen.
\item [Plattform] Eine Plattform ist die Umgebung, in welcher ein Stück Software ausgeführt wird, dabei kann es die Hardware, aber auch das Betriebssystem beschreiben, im Folgenden in Bezug auf Rechnerarchitektur bzw. Betriebssystem benutzt.
\item [Framework] Rahmenapplikation die grundsätzliche Strukturen und Funktionen zur Entwicklung von Programmen bereitstellt. Stellt nicht nur Klassen und Funktionen bereit wie eine Library, sondern in der Regel auch die grobe Anwendungsarchitektur.
\item [Modularität] Das Aufteilen eines Ganzen in logische Teilmodule, welche wiederverwendbar sind.
\item [Logging] Aufzeichnen von Daten während eines Vorgangs
\item [MLT-Datei] ShotCut-Dateiformat 
\item [Code Convention] Programmierrichtlinien
\item [Offset] Zeitversatz (hier: in Videos)
\item [Standalone Software] (Eine Software die ohne andere Software funktioniert)

\end{description}
