\documentclass[12pt, xcolor=table]{beamer}


\usepackage[utf8]{inputenc}
\usepackage[ngerman]{babel}
\usepackage{fontspec}
\usepackage{tabularx}
\usepackage{mathtools}

\setsansfont{Roboto}
\setmonofont{Roboto Mono}

\definecolor{myfontcolor}{RGB}{13, 71, 161}
\definecolor{stringcolor}{RGB}{0,150,20}
\definecolor{keywordcolor}{RGB}{0,90,230}
\definecolor{commentcolor}{RGB}{192,32,32}

\title{Review\\Lastenheft}
\author{tj18b}

\beamertemplatenavigationsymbolsempty 
\setbeamercolor{section in toc}{fg=myfontcolor} 
\setbeamercolor{subsection in toc}{fg=myfontcolor}
\setbeamercolor{section}{fg=myfontcolor}
\setbeamercolor{subsection}{fg=myfontcolor}
\setbeamercolor{frametitle}{fg=myfontcolor}
\setbeamercolor{framesubtitle}{fg=myfontcolor}
\setbeamercolor{title}{fg=myfontcolor}
\setbeamertemplate{section in toc shaded}[default][100]
\setbeamertemplate{footline}[text line]{
	\parbox{\linewidth}{
		\footnotesize
		\insertsection\hfill\insertframenumber/\inserttotalframenumber
	}
}

\renewcommand{\baselinestretch}{1.4}

\begin{document}

\newcommand{\subitem}[1]{
    {\setlength\itemindent{12pt} \item[-]#1}
}

\maketitle
	
\frame{\frametitle{Inhalt}\tableofcontents[sections={1-5}]}
\frame{\tableofcontents[sections={6-10}]}

\section{Visionen und Ziele}
\begin{frame}
	\frametitle{Visionen und Ziele}
	\begin{itemize}
	\item Software für NAO-Team HTWK
		\subitem{Auswertung von Videos}
		\subitem{Konfiguration/Bearbeitung der Auswertung}
		\subitem{n Videos und Logs mit festem Dateiformat}
		\subitem{ Synchronisation von n Videos}
	\end{itemize}
\end{frame}

\section{Rahmenbedingungen und Produkteinsatz}
\begin{frame}
	\frametitle{Anwendungsbereich \& Zielgruppe}
	\begin{itemize}
		\item Synchronisation und Auswertung, also nach einem Turnier
		\item Zielgruppe ist ein Roboterfußballteam
		\subitem{u.A. Studenten, universitäre Mitarbeiter}
	\end{itemize}
\end{frame}

\begin{frame}
	\frametitle{Betriebsbedingungen}
	\begin{itemize}
		\item unregelmäßige Nutzungszeiträume
		\item Software ist eine Desktopanwendung
	\end{itemize}
	\vspace{0.5cm}\textbf{Produktumgebung}\vspace{0.2cm}
	\begin{center}
	\small
	\begin{tabularx}{\framewidth}{lll}
		\textbf{Software}&\textbf{Hardware}&\textbf{Orgware}\\
		Windows/Linux/Java&Aktueller Laptop&Zugriff auf Videodaten\\
	\end{tabularx}
	\end{center}
\end{frame}

\begin{frame}
	\frametitle{Betriebsbedingungen}
	\textbf{Entwicklungsumgebung}\vspace{0.2cm}
	\begin{center}
	\small
	\begin{tabularx}{\framewidth}{lll}
		\textbf{Software}&\textbf{Hardware}&\textbf{Orgware}\\
		Windows/Linux/Mac&PC&Zugriff auf Videodaten\\
		Java \& IDE&&GIT \& Logs\\
	\end{tabularx}
	\end{center}
\end{frame}

\section{Kontext und Überblick}
\begin{frame}
	\frametitle{Kontext und Überblick}
	\begin{itemize}
		\item Standalone-Software
		\item Mobil einsetzbar
		\item Windows und Linux
	\end{itemize}
\end{frame}

\section{Funktionale Anforderungen}
\begin{frame}
	\frametitle{Funktionale Anforderungen}
	\begin{itemize}
		\item paralleles Abspielen von bis zu 4 Videos (+ Bedienelemente)
		\item Synchronisation mit (geparsten) Logfiles
		\item Auswahl der verwendeten Audiospur
		\item Speichern der Bearbeitung in eigenständigem Format(MTL)
		\item Zusatzfeatures wie Schnitt, Audiosynchro und automatischer Kamerawechsel
	\end{itemize}
\end{frame}

\section{Nicht-funktionale Anforderungen}
\begin{frame}
	\frametitle{Nicht-funktionale Anforderungen}
	\begin{itemize}
		\item Wartbarkeit 
		\subitem{Javadoc vollständig}
		\subitem{Dokumentation/Kommentare englisch}
		\subitem{Code conventions}
		\item Interoperabilität
		\subitem {Windows}
		\subitem {Linux}
		\subitem {(MacOS)}
	\end{itemize}
\end{frame}

\begin{frame}
	\begin{itemize}
		\item Bedienbarkeit
		\subitem{Manual}
		\subitem{nutzerfreundliche GUI}
		\subitem{Anname: DAU}
	\end{itemize}
\end{frame}

\section{Produktdaten}
\begin{frame}
	\frametitle{Produktdaten}
	\begin{itemize}
		\item Inputdaten
		\subitem{n Videodaten}
		\subitem{GC Log}
		\item Videodaten
		\subitem{Datei im MP4 Format}
		\subitem{30 FPS}
		\subitem{Offset}
	\end{itemize}
\end{frame}

\begin{frame}
	\begin{itemize}
		\item GC Log
		\subitem{Log-Datei als Textdokument}
		\subitem{Daten zum Spielablauf}
		\item Konfigurationsdaten
		\subitem{automatisierte Prozesse}
		\subitem{Videoformat}
		\item Outputdaten
		\subitem{Video im angegeben Format}
		\subitem{MLT Datei}
	\end{itemize}
\end{frame}

\section{Qualitätsmatrix nach ISO 25010}
\begin{frame}
	\frametitle{Qualitätsmatrix nach ISO 25010}
	\begin{center}
	\small
	\begin{tabularx}{\framewidth}{X|cccc}
		Wichtigkeit&Hoch&Mittel&Niedrig&N.A.\\
		\hline Funktionale Tauglichkeit&X&&& \\
		Effizienz&X&&& \\
		Kompatibilität&&X&& \\
		Nutzbarkeit&&X&& \\
		Zuverlässigkeit&&X&& \\
		Sicherheit&&&&X \\
		Wartbarkeit&X&&& \\
		Übertragbarkeit&X&&& \\
	\end{tabularx}
	\end{center}
\end{frame}

\section{Lieferumfang}
\begin{frame}
	\frametitle{Lieferumfang}
	\begin{itemize}
		\item GUI basiertes Modul 
		\subitem{Video- \& Log-Files einlesen, anzeigen, bearbeiten, darstellen}
		\subitem{Synchronisation Video mit Log}
		\item Dokumentation
	\end{itemize}
\end{frame}

\section{Abnahmekriterien}
\begin{frame}
	\frametitle{Abnahmekriterien}
	\begin{itemize}
		\item Einhalten des Lieferumfangs
		\item optionale Anforderungen
        \end{itemize}
\end{frame}

\section{Vorprojekt}
\begin{frame}
	\frametitle{Vorprojekt}
		\begin{itemize}
		\item Grundger{\"u}st f{\"u}r das weitere Vorgehen
		\item Gui-Entwurf
		\item Synchrone Wiedergabe
		\item Log-Files parsen
        \end{itemize}
\end{frame}

\end{document}