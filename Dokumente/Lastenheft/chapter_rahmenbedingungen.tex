\chapter{Rahmenbedingungen und Produkteinsatz}
\section{Anwendungsbereich \& Zielgruppe} 
Die Software wird während oder nach einem Turnier Verwendung finden. Anwendungsbereich sind somit vom Turnierveranstalter bereitgestellte Räume oder die Örtlichkeiten des Teams.
Die Zielgruppe ist ein Roboterfußballteam bzw. dessen Mitglieder. Mitglieder des Teams sind Studenten, universitäre Mitarbeiter, beziehungsweise deren Professoren.
\section{Betriebsbedingungen}
Das Produkt muss im mobilen sowie stationären Betrieb laufen.
Die Software unterliegt eher unregelmäßigen Nutzungszeiträumen.
\section{Technische Produktumgebung}
Die Software ist eine Desktopanwendung.
\begin{center}
 \begin{tabular}{p{4cm}p{4cm}p{4cm}}
    \textbf{Software} & \textbf{Hardware} & \textbf{Orgware}\\
    Windows/Linux, Linux & Aktueller Laptop & Zugriff auf Videodaten
    \end{tabular}   
\end{center}
\section{Anforderungen an die Entwicklungsumgebung}
\begin{center}
    \begin{tabular}{p{4cm}p{4cm}p{4cm}}
       \textbf{Software} & \textbf{Hardware} & \textbf{Orgware}\\
       Windows/Linux/Mac, Java \& IDE & PC & Zugriff auf Videodaten, Git \& Logs
    \end{tabular}   
\end{center}