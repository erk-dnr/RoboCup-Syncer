\chapter{Aspekte}
\section{Zielsetzung}
Das Ziel unseres Projektes ist die Entwicklung eines Tools zur übersichtlichen Verarbeitung von Roboterfußballvideos und den dazugehörigen Audio- und Log-Files. Dabei soll es möglich sein bis zu 4 Videos sychronisiert nebeneinander darzustellen. Des weiteren soll das Tool die Möglichkeit bieten, eben diese Video- und Audiodateien genaustens zu bearbeiten, insbesondere soll ein Frame- genauer Zugriff auf die Videos möglich sein. 
\section{Rahmenbedingungen}
Den Rahmen für unser Projekt bilden verschiedene Vorgaben der Projektleiter, sowie einige Annahmen, welche getroffen werden, um sonst kritische Fälle zumindest für die Soll- Erfüllung ausschließen zu können. Es liegen GC-Logfiles und Videodateien vor. Bezüglich der Logfiles ist es wichtig, dass die zwei vorliegenden Files geladen werden können müssen. Die Videodateien sind alle im gleichen Format und haben die gleiche FPS-Zahl. Des weiteren ist der Offset der Videos zum Start des Spieles bekannt.
\section{GUI}
Da sich unser Projekt um die Bearbeitung von mehreren synchronisierten Videos dreht, ist ein intuitives und übersichtliches GUI entscheidend. Die Funktionalität und die Features bilden zwar einen wichtigen Bestandteil, dennoch ist es wichtig das GUI möglichst einfach und vor allem klar strukturiert aufzubauen, um die Übersichtlichkeit zu gewährleisten. Durch den Umstand, dass das entstehende Tool einer spezifischen Gruppe zur Verfügung gestellt wird, ist es uns möglich durch regelmäßige Rücksprache die Wünsche der Zielgruppe soweit umsetzbar zu implementieren.
\section{Problemfälle}
Im späteren Projektverlauf wird die Behandlung der in den Rahmenbedingungen bereits erwähnten Problemfälle möglicherweise ein bedeutender Aspekt. Es ist zu betrachten, wie man mit Änderungen des GC-Formates umgehen kann, da für dieses keine Spezifikation vorliegt. Auch gesplitete Log-Dateien, eine bruchstückhafte oder sogar fehlende Audiospur, unterschiedliche Videoformate oder verschiedene Fps-Zahlen der Videos, sowie fehlende Informationen bezüglich der Offsets sind zu betrachten.
\section{Ansatz}
Die Wahl der geeigneten Libary ist ein wesentlicher Bestandteil für den Erfolg des Projektes. Darum ist es wichtig, sich im Rahmen der Recherche ausführlich mit den zur Verfügung stehenden Liberys auseinanderzusetzen. Im Idealfall ist die gewählte Libery intuitiv bedienbar, modularisierbar und erfüllt sämtliche Funktionalitäten, welche im Rahmen des Projektes verlangt werden. Bei der Auswertung der Konzepte ist aufgefallen, dass sowohl VLcJ als auch OpenCV unseren Ansprüchen nicht vollends gerecht werden. VLcJ bietet keine Möglichkeiten der Video- und Audiobearbeitung, des weiteren ist eine multiple Einbettung von Videos nicht möglich, somit besitzt das Package essentielle Funktionalitäten nicht und ist daher für unser Projekt eher ungeeignet. OpenCV bietet umfangreiche Möglichkeiten der Videobearbeitung, jedoch fehlt die Audio-Unterstützung. Es besteht die Möglichkeit, dies durch ein anderes Package zu kompensieren. Xuggler bietet die geforderten Funktionalitäten in vollem Umfang hinsichtlich der Video- und Audiobearbeitung.
