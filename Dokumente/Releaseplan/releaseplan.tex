\chapter{Releaseplan}

\section{Einleitung}

Das Projekt wird in fünf Releasezyklen umgesetzt. Im folgenden Releaseplan ist festgehalten, welche Features in welchem Release umgesetzt werden sollen. 
Dabei werden für den aktuellen Releasezyklus die Aufgaben aus dem Releaseplan in Tasks heruntergebrochen, als Issues aufbereitet und konkreten Personen im Team zur Bearbeitung zugewiesen. 
Zur jeweiligen Abgabe ist zu prüfen, welche Aufgaben erfüllt wurden und nicht fertiggestellte Tasks neu einzuordnen durch die Aktualisierung des Releaseplanes.

\section{Erstes Release, Abgabe: 07.01.19}

Im ersten Release geht es um die Umsetzung des bereits beschriebenen Vorprojektes. 
So ist das Ziel, ein Grundgerüst der im Projektplan beschriebenen Arbeitspakete für das weitere Vorgehen zu schaffen. 
Ein vollständiges Klassengerüst ist erstellt worden auf dem eine weitestgehend noch unfunktionales GUI läuft, aus dem alle gewünschten Bedienelemente und Funktionalitäten ersichtlich werden. 
Neben einer klaren Aufteilung der Hauptelemente wie Videoplayer, Loganzeige und Bearbeitungsfeld für die einzelnen Spuren, sollte eine Menüleiste eingebettet sein und ein grundsätzlicher Designkern erkennbar sein. 
Die Menüleiste enthält zunächst die Punkte Datei -> öffnen, speichern, Videos laden, Logfiles laden, als auch
Bearbeiten -> Rückgängig(eventuell), Bild laden, Audiospur auswählen, Hilfe -> Dokumentation anzeigen, Bedienungsanleitung anzeigen.
Darüber hinaus wird ein Gerüst für die Dokumentation und die Bedienungsanleitung angelegt.

\section{Zweites Release, Abgabe: 21.01.19}

Im Rahmen des Zweiten Releases wird das erstellte Gerüst mit ersten Funktionalitäten ergänzt. 
Die Logs können bereits geparst werden, Videos abgespielt und angehalten werden können. 
Ein entsprechendes GUI-Element zum auswählen der Video und Log-Dateien ist implementiert. 
Alles bereits Erstellte ist sauber dokumentiert. 
Eventuelle Änderungswünsche aus dem ersten Release wurden berücksichtigt. 

\section{Drittes Release, Abgabe: 04.02.19}

Ziel des dritten Releases ist die weitere Ergänzung von Funktionalitäten.
So werden nun beim Laden der Videos auch die entsprechenden Trackpanes für die Spuren erstellt. 
Markierungen können bereits optisch gesetzt werden, bleiben aber noch ohne Funktion oder Wirkung. 
Vier Videos können nun bereits parallel abgespielt, gestoppt, und mit allen gewünschten Sprungoptionen durchlaufen werden. 
Die geparsten Logs sind mit den Videos synchronisiert und können gegenseitig angesteuert werden. 
Auch die Menüleiste verfügt bereits über Ansätze der zu implementierenden Funktionen. 
Die bis hierhin erstellte Dokumentation ist vollständig bezüglich des Funktionsumfanges. 
Selbiges gilt auch für die Bedienungsanleitung.

\section{Viertes Release, Abgabe: 11.03.19}

Im viertem Release wird eine Vorabversion mit vollem Funktionsumfang geliefert. 
MLT-Dateien können aus dem Bearbeitungsprozess erstellt werden. 
Die Markierungen in den Trackpanes erzeugen die entsprechenden Daten und alle Funktionalitäten aus der Menüleiste sind implementiert. 
Die Dokumentation ist bereits vollständig und kann auf eventuelle Verbesserungen geprüft werden. 
Die Bedienungsanleitung ist noch nicht komplett, da eventuelle Änderungswünsche in der Bedienung berücksichtigt werden sollen.

\section{Fünftes Release, Abgabe: 18.03.19}

Der Lieferumfang von dem fünften Release umfasst ein vollständig funktionsfähiges, stabiles und sauber kompilierendes Programm. 
Alle Fehlerquellen wurden entfernt, entsprechendes Exceptionhandling vervollständigt, Unsauberkeiten in der GUI entfernt und das Designkonzept ausgereift umgesetzt. 
Eventuelle Zusatzfeatures wurden vollständig und fehlerfrei, oder gar nicht implementiert. 
Die Dokumentation sowie die Bedienungsanleitung sind übersichtlich, vollständig und fehlerfrei.
