\chapter{Projektplan}
\section{Einführung}
Das Projekt wurde in folgende Arbeitspakete unterteilt: Vorprojekt, Synchronisation, Basisfunktionalität, Dokumentation und Ein-/Ausgabe.\\
Die Einteilung erfolgte mit der Absicht, die Aufgabenverteilung im weiteren Verlauf des Projekts zu vereinfachen und einen groben Ansatz zur geforderten Modularität zu schaffen.

\section{Vorprojekt}
Dieses Paket enthält alle Funktionen, die das Vorprojekt ausmachen.
\subsection{Ziele}
Erstellen eines grundlegenden Klassengerüsts des gesamten Projekts und einer (nicht-funktionalen) GUI. 
Benutzeroberfläche übersichtlich, intuitiv gestalten.\\
Zentralen Bereich zum Abspielen mehrerer Videos, Kontextmenüleiste, teilweise konfigurierbare Bedienelemente, Editierspur, in der Abschnitte markiert werden können und eingelesene GC-Logs zur Nutzerinteraktion bereitstellen.\\

Geschätzter Aufwand: 10\%

\section{Synchronisation}
Dieses Paket enthält Funktionen zur Synchronisation von Videos.

\subsection{Ziele}
Frame-genaues springen, Videos per Offset synchron abspielen und eingelesene GC-Logs parsen.

\textbf{Kann-Ziele}\\
Automatische Kameraauswahl und parsen von Roboter-Logs.\\

Geschätzter Aufwand: 20\% (+15\% Kann-Ziele)

\section{Basisfunktionalität}
Dieses Paket enthält grundlegende Funktionen, auf die andere Paketen aufbauen.
\subsection{Ziele}
Mehrere Videos abspielen, Audiospur eines Videos auswählen, Editiermarkierungen setzen, anhand geparster Logs synchron, mit einstellbarem Zeitabstand vor ein Log-Ereignis springen.

\textbf{Kann-Ziele}\\
Videos zu FPS konvertieren, Overlaygrafik generieren, letzte Aktion rückgängig machen.

Geschätzter Aufwand: 40\% (+15\% Kann-Ziele)

\section{Ein-/Ausgabe}
Dieses Paket enthält Funktionen zum Einlesen/Ausgeben von Dateien.
\subsection{Ziele}
GC-Logs einlesen, Zwischenstand in eigenem Dateiformat speichern und wieder einlesen, Exportieren von Editiermarkierungen in MLT-Datei.

\textbf{Kann-Ziele}\\
Roboter-Logs einlesen, mehrere Videoformate einlesen, Video ohne Audio/nur Audiospur behandeln.

Geschätzter Aufwand: 25\% (+15\% Kann-Ziele)

\section{Dokumentation}
Dieses Paket enthält alle Anforderungen an die Dokumentation
\subsection{Ziele}
Parallel zum Projekt Code kommentieren, JavaDoc führen, Handbuch schreiben.

Geschätzter Aufwand: 5\%