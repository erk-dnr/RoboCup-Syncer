\chapter{Organisatorische Festlegungen}
	Benutzung eines iterativen Entwicklungsmodell; das System wird als Folge von Release-Versionen zu vereinbarten Terminen entwickelt werden, dabei werden die Funktionalitäten Schritt für Schritt erweitert. Dabei werden zu den unten genannten Abgabetermine ein aktuelles Releasebündel bestehend aus Quellcode, Testmaterial, Entwurfsbeschreibung, Testbericht, Releaseplan, Aufwandanalyse sowie eine Demoversion des aktuellen Releases auf der Website.

	Die Abgabedatei ist eine zip-Datei, wobei Quellcode, Beschreibungen und Testmaterial jeweils in eigenen Verzeichnissen abzulegen sind.

	\section{Abgabetermine}
		\begin{center}
			\begin{tabularx}{\textwidth/2}{Xc}
				\textbf{Termin}&\textbf{Releasebündel}\\\hline
				7.01.2019&R1\\
				21.01.2019&R2\\
				04.02.2019&R3\\
				04.03.2019&R4\\
				18.03.2019&R5\\
			\end{tabularx}
		\end{center}

	\section{Validierung}
		Zur Softwarequalitätssicherung ist Validierung unumgänglich, daher ist ein wöchentliches Treffen mit dem Auftraggeber vereinbart zur Validierung der Software. Dabei werden die aktuellen Releasebündel und den aktuellen Stand im Prozess diskutiert. Beginnend mit dem 10. Januar werden die Treffen Donnerstags um 15 Uhr durchgeführt. Dabei stehen folgende Räumlichkeiten zur Verfügung:
		\begin{center}
			\begin{tabularx}{\textwidth/2}{Xc}
				\textbf{Termin}&\textbf{Raum}\\\hline
				31.01.2019&P801\\
				07.03.2019&P702\\
				Sonst&P901\\
			\end{tabularx}
		\end{center}
