\section{Risikoidentifizierung}
Im ersten Schritt unserer Risikoanalyse befassen wir uns mit der Risikoidentifizierung, das heißt alle mögliche Risiken müssen erfasst werden. Anhand von den Softwareanforderungen, den Rahmenbedingungen und anderen Einflussfaktoren fassen wir unsere Projektrisiken auf folgende zusammen:

\vspace{2em}

\begin{description}
	\item [Andere Module]\ \\Da andere Module auch viel Aufmerksamkeit benötigen, könnte dies zu Zeitproblemen führen.
	\item [Zeit verschätzen]\ \\Nach hinten raus, wird dies den Projekterfolg gefährden.
	\item [Team/Erfahrung]\ \\Anlernen benötigt Zeit, die wir vielleicht nicht haben.
	\item [Unterschiedliche Geschwindigkeiten]\ \\Könnte Probleme bei der Arbeitsteilung geben.
	\item [Buildserverprobleme]\ \\Wird problematisch Änderungen zu testen, wenn kein Build gebaut werden kann.
	\item [Anforderungsmanagement]\ \\Aus Falschinterpretation der Anforderungen folgt Gefährdung des Projekts.
	\item [Libraryprobleme]\ \\Mindert die Entwicklungsgeschwindigkeit.
	\item [Krankheit]\ \\Kalte Jahreszeit, könnte zu Zeitverlust führen, trotz 7 Personen.
	\item [Kommunikationsprobleme]\ \\Führt unweigerlich zu Verzögerungen und Missverständnissen.
	\item [Projektressourcen nicht rechtzeitig verfügbar]\ \\Keine Arbeit möglich, Zeitprobleme.
\end{description}

\pagebreak