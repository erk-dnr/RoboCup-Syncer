\section{Risikobewertung}
Im folgenden werden unsere Risiken näher analysiert und bewerten, d.h. es wird zu jedem Risiko eine Eintrittswahrscheinlichkeit und der Schaden beim Eintritt ermittelt.
Da es bei einigen Risiken schwer ist, zu beurteilen wie schwer und wie wahrscheinlich sie eintreten werden, gliedern wir beides wie folgt:

\vspace{2em}

\begin{table}[h]
	\caption*{Tabelle 1: Bewertungsskala}
	\begin{center}
		\begin{tabularx}{\textwidth}{lXl}
			\textbf{Eintrittswahrscheinlichkeit}&&\textbf{Schaden bei Eintritt}\\\hline
			\textit{Wenig wahrscheinlich} <25\%&&\textit{Gering}: wenig rel. Ergebnisse gefährdet\\
			\textit{Wahrscheinlich} 25\%-75\%&&\textit{Mittel}: Teilergebnisse gefährdet\\
			\textit{Sehr wahrscheinlich} >75\%&&\textit{Schwer}: Erfolg des ges. Projektes gefährdet\\
			\textit{Fakt} 100\%&\\\hline
		\end{tabularx}
	\end{center}
\end{table}

\vspace{2em}

\begin{table}[h]
	\caption*{Tabelle 2: Risikobewertung}
	\begin{center}	
		\begin{tabularx}{\textwidth}{lXll}
			\textbf{Risiken}&&\textbf{Schaden}&\textbf{Wahrscheinlichkeit}\\\hline
			(1) Andere Module&&Schwer&Wahrscheinlich\\
			(2) Zeit verschätzen&&Schwer&Wenig wahrscheinlich\\
			(3) Team/Erfahrung&&Gering&Wenig wahrscheinlich\\
			(4) Unterschiedliche Geschwindigkeiten&&Mittel&Wahrscheinlich\\
			(5) Buildserverprobleme&&Gering&Wenig wahrscheinlich\\
			(6) Anforderungsmanagement&&Schwer&Wenig wahrscheinlich\\
			(7) Libraryprobleme&&Mittel&Wahrscheinlich\\
			(8) Krankheit&&Gering&Wahrscheinlich\\
			(9) Kommunikationsprobleme&&Mittel&Wenig wahrscheinlich\\
			(10) Projektress. nicht rechtzeitig verfügbar&&Schwer&Fakt\\\hline
		\end{tabularx}
	\end{center}
\end{table}

\ \\
\newline
Das Zustandekommen der entsprechenden Wahrscheinlichkeiten und Gewichtungen soll im folgenden kurz erläutert werden:
\newline\newline
Da ein beträchtlicher Teil der Gruppenmitglieder ein enormes Aufkommen an anderen Modulen in diesem Semester aufweist, ist es sehr wahrscheinlich, dass es hier zu Kollisionen kommen kann. Der Schaden für das Projekt ist insofern als schwer anzusehen, dass insbesondere das umfangreiche Prüfungsvolumen und eine dadurch notwendige Priorisierung dieser zu Verzögerungen führen kann. 
Auch zeitliche Fehlkalkulationen könnten derartige Verzögerungen auslösen, welche allerdings aufgrund des vorherrschenden Bewusstseins über diese Problematik als eher unwahrscheinlich bewertet werden. Die nicht homogene Zusammensetzung des Teams bezüglich Erfahrung und vorhandener Kenntnisse wurde zwar als mögliche Risikoquelle erkannt, wird aber als nicht sehr wahrscheinlich und schwerwiegend beurteilt, da genug Potential vorhanden sein sollte, eventuelle Schwachstellen aufzufangen. Ähnlich verhält es sich mit der Problematik der unterschiedlichen Arbeitsgeschwindigkeit.
Buildserverprobleme sollten auch bei Auftreten keine allzu großen Folgen für den Erfolg des Projektes haben und werden aufgrund der guten Behebbarkeit als nicht sehr wahrscheinlich eingestuft.
Schwerwiegend wäre hingegen ein falches Anforderungsmanagement, da hier das Ergebnis des Projekts stark beeinträchtigt wäre. Auch hier senkt allerdings bereits die Tatsache der Bewusstwerdung, die Wahrscheinlichkeit für das Auftreten, da dem mit entsprechender Kommunikation entgegengewirkt werden kann.
\newline\newline
Probleme mit externen Librarys und Packages werden erwartet, da dies bislang eine eher unbekannte Größe und schwer vorherzusehen ist. Die Auswirkungen sind als mittel eingestuft, da es sich dabei i.d.R um gut lösbare Probleme handelt. Der Faktor Krankheit sollte eine eher untergeordnete Rolle spielen, da es genug Kapazitäten im Team gibt um kurzfristige Ausfälle abzufangen.
\newline\newline
Eventuelle Kommunikationsprobleme können zwar nicht zu vernachlässigende Auswirkungen auf das Gesamtergebnis haben, sind aber ebenso aufgrund der bewussten Überlegungen zu zielgerichteter Kommunikation unwahrscheinlich.
\newline\newline
Da die Projektressourcen zum Zeitpunkt dieser Analyse nicht für alle Teammitglieder voll zugänglich waren, wurde dies bereits als Fakt aufgeführt. Gravierendere Probleme in dieser Hinsicht könnten den Erfolg des Projektes gefährden.


\pagebreak
