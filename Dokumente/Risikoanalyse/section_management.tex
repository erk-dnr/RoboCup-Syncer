\section{Risikomanagement}
Zum Abschluss unserer Risikoanalyse identifizieren wir die Ursachen unserer Risiken und entwickeln Maßnahmen um die Risiken zu minimieren.

\vspace{2em}

\begin{table}[h]
	\caption*{Tabelle 3: Risikomanagement}
	\begin{center}	
		\begin{tabularx}{\textwidth}{clXl}
			\textbf{Risiko}&\textbf{Ursache}&&\textbf{Maßnahme}\\\hline
			(1)&Mehr Zeitaufwand als angegeben&&Zeitmanagement\\
			(2)&Schlechte Organisation&&Zeitmanagement\\
			(3)&Zufälliges Team&&Kommunikation\\
			(4)&Zufälliges Team&&Kommunikation\\
			(5)&Geringe Erfahrung&&Lernen/Studieren\\
			(6)&Kommunikation&&Agile SWE/Issues\\
			(7)&Schlechte Dokumentation&&Verstehen, eigene Doku.\\
			(8)&Winter&&-\\
			(9)&Mangelndes Kommunikationsbewusstsein&&Kommunizieren/planen/GIT\\
			(10)&Uni/Prüfer&&-\\\hline
		\end{tabularx}
	\end{center}
\end{table}

\ \\
\textbf{Fazit}
Es existieren mehrere bedrohliche Risiken, die potentiell Projektgefährdent sind wie "Andere Module" und "Zeit verschätzen", die jedoch mit Zeitmangement und Kommunikation in unsere Projektgruppe zu kontrollieren sind, sodass besonders darauf geachtet werden muss, dass ein regelmäßiger Informationsaustusch bei den Projektteilnehmern gewährleistet ist. Risiken wie "Krankheit" und "Projektressourcen stehen nicht rechtzeitig zur Verfügung" sind nicht bzw. nur teilweise durch das Projektteam beeinflussbar. Insgesamt ist aber zu sagen, dass das Projekt voraussichtlich abgeschlossen werden kann.


