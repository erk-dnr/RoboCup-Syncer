\chapter{Testkonzept}
\section{Einführung}
Das Testkonzept berücksichtigt die Qualität der einzelnen Teile Ihrer Entwicklung (Komponententests)
sowie die Qualität der Zusammenführung der Teile (Integrations- und Systemtest).\\
Hierbei halten wir uns grob an die IEEE829 (Testnorm) und an das Continuous Integration Prinzip.

\setlength{\itemsep}{-2pt} 
\section{Zu testende Eigenschaften}
\begin{itemize}
	\item Komponententests
	\subitem Videopane-Komponente, Log-Parsing, Setzen der Marker, Laden eines Videos
	\subitem Konfigurationsdateien, Schreiben der .mlt-Dateien, Tonspur, optionale Features
	\subitem Schaltflächen
	\item Integrationstests
	\subitem Zusammenarbeit der Komponenten (z.B. Tonspur und VideoPane)
	\subitem Synchronisation
	\subitem Parallele Wiedergabe mit Offsets
	\item Systemtest
\end{itemize}
\setlength{\itemsep}{0pt} 

\section{Testansatz/-strategie}
Grundsätzlich ist jede/s  Klasse/Funktion/Feature der Software zu testen.\\
Zu jedem abgeschlossenen Issue wird ein Issue für dessen Tests angelegt.\\
Am Namen der Tests muss klar erkennbar sein, was getestet wird.

Um eine geordnetes Projekt zu gewährleisten, wird eine Ordnerstruktur angelegt, 
welche das Zuordnen der Tests vereinfachen soll. \\
Diese folgt folgendem Schema:
\begin{center}
	Package\rightarrow Ordner (z.B. Test)\rightarrow Ordner (z.B. Videopane)\rightarrow\dots\rightarrow KlassennameTest.java
\end{center}

Beim Bauen des Projekts sollen die Tests durchlaufen.
Ein Test ist bestanden, wenn die zutestende Funktion den Anforderungen entspricht.

\section{Zuständigkeiten}
Die Testklasse wird von einem Teammitglied geschrieben, welches nicht die Klasse geschrieben hat. Hiermit soll
gleichzeitig getestet werden, ob der Code verständlich geschrieben ist.
Das Anlegen der Testissues obliegt dem Bearbeiter des Features.

\section{Testrelevanz}
Testrelevanz hat jede Funktion, welche Werte ausgibt, die in der Funktion bearbeitet wurden
(Simple Getter-Funktionen müssen z.B. nicht getestet werden).

\section{Risiken}
Ein Risiko ist das planlose Durchführen der Tests. 
Außerdem ist der Zeitfaktor beim Schreiben der Tests kritisch.
Deshalb ist das Schreiben der Tests im Fall der Fälle nieder priorisiert als das Schreiben der Features.
Es muss darauf geachtet werden, dass Funktionen, die klar keine Testrelevanz haben, ausgelassen werden.

